\section{Related Works}
\label{sec1}

Some scholars found that current boolean search and filtering techniques cannot satisfy the complexity of candidate-job matching requirement~\cite{malinowski2006matching}. They hope the system can understand the job requirement, determine which requirements are mandatory and which are optional, but preferable. So they moved to use recommender systems technique to address the problem of information overflow. Recommender systems are broadly accepted in various areas to suggest products, services, and information items to latent customers.

\subsection{Recommender System}
\label{subsec1}

Job searching, which has been the focus of some commercial job finding web sites and research papers is not a new topic in information retrieval. Usually scholars called them Job Recommender Systems (JRS), because most of them used technologies from recommender systems. Wei et al. classified Recommender Systems into four categories~\cite{wei2007survey}:

\begin{enumerate}
    \item Content-based Recommendation (CBR). The principle of a content-based recommendation is to suggest items that have similar content information to the corresponding users, like Prospect \cite{singh2010prospect}.

    \item Collaborative Filtering Recommendation (CFR). Collaborative filtering recommendation finds similar  users  who have  the same taste with the target user and recommends items based on what the similar users, like CASPER~\cite{rafter2000personalised}.

    \item Knowledge-based Recommendation (KBR). In the knowledge-based recommendation, rules and patterns obtained from the functional knowledge of how a specific item  meets the requirement of a particular user, are used for recommending items, like  Proactive~\cite{lee2007fighting}.

    \item Hybrid recommender systems combine two or more recommendation techniques to gain better performance, and overcome the drawbacks of any individual one. Usually, collaborative filtering is combined with some other technique in an attempt to avoid the ramp-up problem.

\end{enumerate}

\subsection{Job Recommender System}

Rafter et al. began to use Automated Collaborative Filtering (ACF) in their Job Recommender System, ``CASPER''  \cite{rafter2000personalised}. The system use cluster-based collaborative filtering strategy. The similarity between users are based on how many jobs they both reviewed, or applied. F{\"a}rber et al. \cite{farbr2003automated} presented a recommender system built on a hybrid approach. The system integrate two methods, content-based filtering and collaborative filtering, which tries to overcome the problem of rating data sparsity by leveraging a combined model. In the system, the data source is synthetic resumes. 

\subsection{Information Extraction in Job Recommender System}

Big IT companies met the similar problem of information overflow when they received many resumes for one job opening. The recruiter had to screen the all the applications manually, but this task was also tedious and time consuming. For this reason these companies tried to build systems to help screen the resumes.

Amit et al. in IBM presented a system, ``PROSPECT''~\cite{singh2010prospect}, to aid shortlisting candidates for jobs. The system uses a r\'esum\'e miner to extract the information from r\'esum\'es, which use a conditional random field (CRF) model to segment and label the r\'esum\'es. HP also built a system to solve the similar problem, which is introduced in Gonzalez et al.'s paper~\cite{gonzalez2012adaptive}. The system uses a layered information extraction framework to processing r\'esum\'es. The goal of the systems built by IBM and HP is to help the companies to select good applicants, but cannot help job seekers to find appropriate jobs.

Yu et al.~\cite{yu2005resume} used a cascaded IE framework to get the detailed information from the r\'esum\'es. In the first stage, the Hidden Markov Modeling (HMM) model is used to segment the r\'esum\'e into consecutive blocks. Based on the result, a SVM model is used to obtain the detailed information in the certain block, the information include: name, address, education etc. Celik Duygua and Elci Atilla proposed a Ontology-based R��sum�� Parser (ORP)~\cite{ccelik2013ontology}, which uses ontology to assistant the information extraction process. The system processes a r\'esum\'e in following steps: converting the r\'esum\'e file into plain text, separating the text into some segments, using the ontology knowledge base to find the concepts in the sentences, normalizing all the terms, and classifing the sentences to get the wanted terms.

\subsection{Matching Algorithms in Job Recommender Systems}

Lu et al.~\cite{lu2013recommender} used latent semantic analysis (LSA) to calculate similarities between jobs and candidates, but they only selected two factors ``interest'' and ``education''  to compare candidates. Xing et al.~\cite{yi2007matching} used structured relevance models (SRM) to  match r\'esum\'es and jobs. Drigas et al.\cite{drigas2004expert}  presented a expert system to match jobs and job seekers, and to recommend unemployed to the positions. The expert system used Neuro-Fuzzy rules to evaluate the matching between user profiles and job openings. Daramola et al.\cite{daramola2010fuzzy}  also proposed a fuzzy logic based expert system(FES) tool for online personnel recruitment. In the paper, the authors assumed that the information already be collected. The system uses a fuzzy distance metric to rank candidates' profiles in the order of their eligibility for the job.

