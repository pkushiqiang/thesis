\chapter{MODEL SIMILARITY}

In our system, the r\'esum\'e model and job model both have four fields: job title, major, academic degree and skills. The similarity value between a r\'esum\'e model and job model is the sum of the productions of similarity values of all the fields pairs and their weights. We will introduce how to calculate the similarity value for each field in this chapter.

\section{Similarity of Major and Academic Degree}

In the simplest case, if the majors in the r\'esum\'e model and job model are the same, the similarity value is 1. If they are different, we can check whether the major in the r\'esum\'e model is in the list of related majors for the major in the job model. If it is, the similarity value is 0.5; otherwise the similarity value is 0. The equation is shown below:

$$ MajorSim(r,j ) = \begin{Bmatrix}
1, & r_{major} = j_{major} \\
0.5, & r_{major} \in related( j_{major} ) \\
0, & otherwise
\end{Bmatrix} $$

There are five kinds of academic degrees in the system:  high school, associate, bachelor, master, and Ph.D., which are mapped to the integer values form 1 to 5. If the degree value in the r\'esum\'e model is less than that in the job model, which means that the job seeker's education background cannot satisfy the requirement of the job, the similarity value in this case is 0. If the degree value in the r\'esum\'e model is equal to the job model and no more than 2 above, the similarity value is 1. In some cases, the degree value in the r\'esum\'e model is greater than that of the job model, and the difference is greater than 2, which means that the job seeker's degree is much higher than the requirement of the job. The situation is also a kind of relative matching, so the similarity value here is 0.5. The equation is shown below:

$$ DegreeSim(r,j ) = \begin{Bmatrix}
0,   & r_{degree} < j_{degree} \\
1,   & 0 < r_{degree} - j_{degree} \leqslant 2  \\
0.5, & r_{degree} - j_{degree} > 2
\end{Bmatrix} $$

\section{Similarity of Job Title}
Another field of needs similarity calculation is the job titles in the models. A job title can be parsed into some sub fields: job role, level, platform, programming language.  The value of job roles includes: developer, manager, administrator and so on. There are levels values in such roles: such as junior, senior and architect. The platforms: web, mobile and cloud are used by the very roles.  The similarity value between two titles is the sum of all the similarity values of these fields. If the job seeker has some working experience, there may be some job titles in their r\'esum\'e.  When calculating the similarity value between a r\'esum\'e model and a job model, the system calculates the similarity values of the title of job model to all the titles in the r\'esum\'e model and returns the maximum one.

We introduced how to calculate similarity values for three fields in r\'esum\'e and job models. In the next chapter, we will introduce how to use a domain specific ontology to calculate similarity values between the skills fields of the two models.

