\begingroup
\absone
{JOBFINDER: A PERSONALIZED RESUME-JOB MATCHING SYSTEM}
{October 2014}
{Shiqiang Guo}
{B.S., Dalian Maritime University}  % Degrees ALREADY RECEIVED,
                       % e.g. {B.S., Rice University;\\
                       % M.S., Texas A\&M University}
                       % If only one degree, delete `;\\'
{Dr. Tracy Hammond}%put your advisor name here
{
Today, online recruiting web sites such as Monster and Indeed have become one of the main channels for people to find jobs. These web platforms have provided their services for more than ten years, and have saved a lot of time and cost for both job seekers and organizations who want to hire people. However, traditional information retrieval techniques may not be appropriate for users. The reason is because the number of results returned to a job seeker may be huge, so job seekers are required to spend a significant amount of time reading and reviewing their options. One popular approach to resolve this difficulty for users are recommender systems, which is a technology that has been studied for a long time.

In this thesis we have made an effort to propose a personalized job-r\'esum\'e matching system, which could help job seekers to find appropriate jobs more easily.  We create a finite state transducer based information extraction library to extract models from r\'esum\'es and job descriptions. We devised a new Statistical-based ontology similarity measure to compare the r\'esum\'e models and the job models. Since the most appropriate jobs will be returned in first, the users of the system may get a better result than current job finding website. To evaluate the system, we compared Normalized Discounted Cumulative Gain (NDCG) of the job searching results of the system to other classic information retrieval approaches, and the result showed that the system has  qqqq advantage over current approaches.}
\endgroup



%\abstwo
%{First Line of Title\\Second Line of Title}
%{Month Year}
%{Your Full Name}
%{Degree, University;\\Degree, University}
%{Co-Chair's Name}
%{Co-Chair's Name}
%{Place your abstract between these braces. The text of your abstract must not
%exceed 350 words. Place your abstract between these braces. The text of your
%abstract must not exceed 350 words.}
