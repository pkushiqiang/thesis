
\section{User Study}

We evaluated the system by the user study as well. The purpose of the evaluation is to test the hypothesis that r\'esum\'e-job matching approach will return better results than keywords searching approach.

\subsection{Methodology of User Study}

Ten users participated user study, they are graduate and undergraduate students of Computer Science department of Texas A\&M University. The basic requirement of the users is that they can understand the meanings of technical terms of job description.

At the start of the study, the experimenter introduced the system to the users. He introduced the features of the system, and gave a demo of how to use the three different methods in the system to search the jobs. Then a subject was given a r\'esum\'e, which was selected from the r\'esum\'es downloaded from the internet randomly. The user spent approximately 10 minutes to familiar themself with it, and got initial idea of what jobs are appropriate to the r\'esum\'e. The subject was asked to search a specific kind of job, like web developer or software engineer with the three methods. During the searching process, the time they used to search and the number of jobs they had reviewed for each method were recorded.  At the end of the study, the subject was asked to take a survey that asked  the personal judgment about the results accuracy of the three methods.


\subsection{Results}

The results of the study are shown in Table~\ref{tab:methodcompare}. We can see from the result that the r\'esum\'e-job matching approach the hybrid approach used shorter time to search the jobs, used the shortest time to search the jobs, and keyword searching used the longest time.


\begin{table}[ht]
\caption{Comparatione of Three Searching Models } % title of Table
\centering % used for centering
\begin{tabular}{  | c | c | c | c | }
 \hline
 Method                    &  Time Used(Minutes)    & Job Reviewed & Accuracy Score  \\
 \hline
 Keyword                   & 6.3                    & 15.8         &       3.2         \\
 \hline
 Resume Matching           & 5.2                    & 14.6         &       4.1         \\
  \hline
 Keyword + Resume Matching & 4.3                    & 13.2         &       4.1       \\
  \hline
\end{tabular}
\label{tab:methodcompare} % is used to refer this table in the text\section{Pipeline of Information Extraction}
\end{table}

